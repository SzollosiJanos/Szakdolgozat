\Chapter{Bevezetés}
A szakdolgozat célja egy olyan függvénykönyvtár létrehozása, amely automatikusan létrehoz egy beimportált 3 dimenziós modellhez egy "hitboxot", amely alapján kiszámíthatjuk az adott modell ütközését más modellekkel. A program \textbf{C} nyelvben íródott, az \textbf{OpenGL} és \textbf{SDL2} függvénykönyvtárak segítségével. A program fő célja a modellek felbontása atomi szintre (háromszögekre), majd a háromszögek metszéspontjának kiszámítása.

A függvénykönyvtár tökéletes lehet kisebb játékok fejlesztéséhez, vagy egyéb fizikai szimulációk motorjaként. Mivel az ütközésvizsgálat atomi szinten történik a modell háromszögekre bontásával, így nagy az ütközésvizsgálat pontossága.
