\Chapter{Koncepció}

A feladat fő problémája a háromszögek metszéspontjának kiszámítása a térben. Ez sajnos nem egy egyszerű feladat. Programozás terén, illetve erőforrásigény terén sem elhanyagolható.

A valóságban az emberi gondolkodásnak egyszerűnek tűnhet eldönteni, hogy két háromszög metszi-e egymást, vagy sem. Programozás, illetve matematika terén viszont sokkal nehezebb. Rengeteg számítást kell végeznünk ahhoz, hogy megbizonyosodjunk a háromszögek metszéséről.

\Section{Irodalomkutatás}
A metszéspontok számításhoz legmegfelelőbbnek \textbf{\href{https://www.geometrictools.com/Documentation/DynamicCollisionDetection.pdf}{David Eberly 1999-es kutatását}}, azon belül a \textbf{4.1-es (Separation of Triangles)} szekciót találtam. A dokumentáció tökéletesen elmagyarázza a matematikai képletek elemeit, azok használatát, lépéseit. Ezek mind táblázatba szedve találhatóak. A pontos magyarázat a következő fejezetben található. Emellett rengeteg különböző módszer található az interneten térbeli modellek ütközésének vizsgálatához, például:\\

\textbullet\textbf{Négyzetes elválasztás:} Lényege, hogy ha két test nem ütközik egymással, akkor a két test között áthaladva létezik egy olyan sík, amely nem metszi őket. Konvex modellek esetén hatékony. \\

\textbullet\textbf{Sugárkövetés:} Lényege, hogy minden modellhez egy halmazt rendelnek, amely tartalmazza a modellt felépítő pontokat. Az átfedő pontokkal vizsgálhatjuk az ütközést. \\

\textbullet\textbf{Voxel-alapú:} Lényege, hogy a térbeli modellt voxelek halmazaként írjuk le. Az ütközést a voxelek átfedésével vizsgálhatjuk. Bonyolult formájú modellek esetén hatékony. \\

\textbullet\textbf{Egyenletek alkalmazása:} Lényege, hogy néhány modell (például: gömbök, síkok, hengerek) már tartalmazza az ütközésvizsgálathoz szükséges információkat. Ilyen modellek esetén egyenleteket használunk az ütközésvizsgálat detektálására. \\