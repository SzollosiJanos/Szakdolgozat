\Chapter{Ütközések számítása}
\label{chap:utkozesek}
A háromszöget metszéspontjának számításához elsősorban szükségünk van 2 háromszögre, mint input. Ezek csúcspontjait jelöljük az $\textbf{A}_0$, $\textbf{A}_1$, $\textbf{A}_2$, illetve $\textbf{B}_0$, $\textbf{B}_1$, $\textbf{B}_2$ pontokkal. A csúcspontok segítségével kiszámíthatjuk a háromszögek éleit. Ezek lesznek a $\textbf{C}_0$, $\textbf{C}_1$, $\textbf{C}_2$, illetve $\textbf{E}_0$, $\textbf{E}_1$, $\textbf{E}_2$ élek: \\
$$\textbf{C}_0 = \textbf{A}_1 - \textbf{A}_0, \textbf{C}_1 = \textbf{A}_2 - \textbf{A}_0, \textbf{C}_2 = \textbf{C}_1 - \textbf{C}_0 .$$\\
Az élek segítségével kiszámíthatjuk a háromszögek normál vektorait. Ezek lesznek a \textbf{D}, illetve \textbf{F} vektorok:  \\
$$\textbf{D = C}_0 \times \textbf{C}_1,\textbf{F = E}_0 \times \textbf{E}_1.$$\\
A számítások megkönnyítéséhez kiszámítjuk az eltolásvektort is:  \\
$$\textbf{G = B}_0 -\textbf{A}_0.$$\\
Ezen adatok szolgálják az alapokat, amelyekből további számításokat végzünk. A metszés eldöntéséhez a következő univerzális képletet használjuk:\\
\hbox to 6cm{}	$\textbf{H}_1 = \textbf{L * C}_0,$\\
\hbox to 6cm{}	$\textbf{H}_2 = \textbf{L * C}_1,$\\
\hbox to 6cm{}	$\textbf{I}_0 = \textbf{L * G},$\\
\hbox to 6cm{}	$\textbf{I}_1 = \textbf{I}_0 + \textbf{L * E}_0,$\\
\hbox to 6cm{}	$\textbf{I}_2 =\textbf{ I}_0 + \textbf{L * E}_1.$\\

Minden sor 1-1 számítást jelent. Minden számítás után ellenőriznünk kell, hogy a két adott háromszög metszi-e egymást, vagy sem. Erre a következő képletet használjuk: Ha\\
$$\textbf{min(H) > max(I) vagy max(H) < min(I)},$$akkor a két háromszögre biztosan mondható, hogy nem metszik egymást. Ez esetben nem szükséges további számításokat végezni. Amennyiben nemleges választ kapunk, akkor tovább kell számítanunk minden sort. Ha az utolsó sor esetén sem kapunk pozitív választ, akkor kimondhatjuk, hogy a két háromszög metszi egymást. A pontos matematikai számításokat a \ref{tab:szamitas} táblázat mutatja.

\newpage

\begin{table}
	\centering
	
	\begin{tabular}{|c||c|c|c|c|c|}
		\hline
		\textbf{L}     & $\textbf{H}_1$   & $\textbf{H}_2$    & $\textbf{I}_0$      & $\textbf{I}_1$        & $\textbf{I}_2$        \\ \hline
		\textbf{D}     & 0    & 0     & D*G     & I0 + D*$E_0$ & I0 + D*$E_1$ \\ \hline
		\textbf{F}     & F*$C_0$ & F*$C_1$  & F*G     & $I_0$        & $I_0$        \\ \hline
		$\textbf{C}_0$*$\textbf{E}_0$ & 0    & -D*$E_0$ & $C_0$$\times$$E_0$*G & $I_0$        & $I_0$ + F*$C_0$ \\ \hline
		$\textbf{C}_0$*$\textbf{E}_1$ & 0    & -D*$E_1$ & $C_0$$\times$$E_1$*G & $I_0$ - F*$C_0$ & $I_0$        \\ \hline
		$\textbf{C}_0$*$\textbf{E}_2$ & 0    & -D*$E_2$ & $C_0$$\times$$E_2$*G & $I_0$ - F*$C_0$ & $I_1$        \\ \hline
		$\textbf{C}_1$*$\textbf{E}_0$ & D*$E_0$ & 0     & $C_1$$\times$$E_0$*G & $I_0$        & $I_0$ + F*$C_1$ \\ \hline
		$\textbf{C}_1$*$\textbf{E}_1$ & D*$E_1$ & 0     & $C_1$$\times$$E_1$*G & $I_0$ - F*$C_1$ & $I_0$        \\ \hline
		$\textbf{C}_1$*$\textbf{E}_2$ & D*$E_2$ & 0     & $C_1$$\times$$E_2$*G & $I_0$ - F*$C_1$ & $I_1$        \\ \hline
		$\textbf{C}_2$*$\textbf{E}_0$ & D*$E_0$ & $H_1$    & $C_2$$\times$$E_0$*G & $I_0$        & $I_0$ + F*$C_2$ \\ \hline
		$\textbf{C}_2$*$\textbf{E}_1$ & D*$E_1$ & $H_1$    & $C_2$$\times$$E_1$*G & $I_0$ - F*$C_2$ & $I_0$        \\ \hline
		$\textbf{C}_2$*$\textbf{E}_2$ & D*$E_2$ & $H_1$   & $C_2$$\times$$E_2$*G & $I_0$ - F*$C_2$ & $I_1$        \\ \hline
	\end{tabular}
	\caption{A számítások táblázata.}
	\label{tab:szamitas}
\end{table}

\newpage

\Section{Függvénykönyvtár tervezése}
A függvénykönyvtár C\cite{C} nyelven íródott. Ez egy egyszerűen megtanulható, gyors programozási nyelv. Létrehozhatunk benne függvényeket, új típusokat/struktúrákat, amelyet később bárhol használhatunk a program készítése során. A készített függvénykönyvtár könnyen használható, mindössze bele kell raknunk az általunk készített program forráskódjai közé a collision\_triangle.c, illetve collision\_triangle.h fájlokat, majd beimportálni azt a forráskódban a '\#include "collision\_triangle.h"' sor segítségével. Ha mindent helyesen csináltunk, akkor máris használhatóak a függvények, illetve struktúrák, amelyeket a \ref{tab:strukt}, illetve \ref{tab:fug} táblázat mutatja be.


\begin{table}[h]
	\centering
	\begin{tabular}{|c|c|c|}
		\hline
		\textbf{Változó}     & \textbf{Típusa}& \textbf{Leírása}\\ \hline
		x&float&Egy háromdimenziós pont helyét jelöli az x tengelyen.\\ \hline
		y&float&Egy háromdimenziós pont helyét jelöli az y tengelyen.\\ \hline
		z&float&Egy háromdimenziós pont helyét jelöli az z tengelyen.\\ \hline
		
	\end{tabular}
	\caption{Vec3 struktúra bemutatása}
	\label{tab:strukt}
\end{table}

\begin{table}[h]
	\centering
	\begin{tabular}{|c|c|c|c|}
		\hline
		\textbf{Függvény} & \textbf{Visszatérése}    & \textbf{Paraméterei}& \textbf{Leírása}\\ \hline
		
		check\_collision&Boolean&\parbox{2.7cm}{\centering modell1\\modell2\\modell1 pozíció\\modell2 pozíció}&\parbox{4.3cm}{\centering A modellek és a modellek pozícióinak megadásával visszakapjuk, hogy ütköznek-e vagy sem.}\\ \hline
		
		sub&Vec3&\parbox{2.7cm}{\centering 2 térbeli pont\\(A,B)}&\parbox{4.3cm}{\centering Visszaadja a két pont különbségét.}\\ \hline
		
		cross&Vec3&\parbox{2.7cm}{\centering 2 térbeli pont\\(A,B)}&\parbox{4.3cm}{\centering Visszaadja a két pont Descartes szorzatát.}\\ \hline
		
		dot&Float&\parbox{2.7cm}{\centering 2 térbeli pont\\(A,B)}&\parbox{4.3cm}{\centering Visszaadja a 2 pont koordinátáinak szorzatát\\(A.x * B.x + A.y * B.y + A.z * B.z)}\\ \hline
		
		getmin&Float&\parbox{2.7cm}{\centering 3 float változó\\(a,b,c)}&\parbox{4.3cm}{\centering Visszaadja a 3 változó közül a legkisebbet.}\\ \hline
		
		getmax&Float&\parbox{2.7cm}{\centering 3 float változó\\(a,b,c)}&\parbox{4.3cm}{\centering Visszaadja a 3 változó közül a legnagyobbat.}\\ \hline
		
		get\_distance&Float&\parbox{2.7cm}{\centering 2 térbeli pont\\(A,B)}&\parbox{4.3cm}{\centering Visszaadja a 2 pont közötti távolságot.}\\ \hline	
			
		scale\_model&Void&\parbox{2.7cm}{\centering modell\\3 float változó a méretezéshez}&\parbox{4.3cm}{\centering Átméretezi az adott modellt.}\\ \hline	
			
		rotate\_modele&Void&\parbox{2.7cm}{\centering modell\\3 float változó a forgatáshoz}&\parbox{4.3cm}{\centering Forgatja a térben a modellt.}\\ \hline		
		
		mirror\_model&Void&\parbox{2.7cm}{\centering modell\\tengely indexe\\0=z,1=y,2=x}&\parbox{4.3cm}{\centering Tükrözi a modellt a megadott tengelyre.}\\ \hline
		
	\end{tabular}
	\caption{Függvények bemutatása}
	\label{tab:fug}
\end{table}
