\Chapter{Összefoglalás}

A szakdolgozat eredménye egy könnyen használható függvénykönyvtár, amely lehetővé teszi 3 dimenziós modellek beimportálását, valamint "hitbox"\textsuperscript{\ref{def:hitbox}}-ként kezelését, ütközések vizsgálatát más modellekkel. A program fejlesztése során magas figyelmet kapott a felhasználóbarát megoldás, így az alkalmazást könnyen használhatják más fejlesztők, akik nem jártasak, vagy csak kevés tapasztalattal rendelkeznek ütközések vizsgálatában.\\

A beépített funkciók azonnal elérhetővé válnak, amint beimportáljuk a függvénykönyvtárat a fejlesztői környezetbe. Fontos megemlítenünk, hogy a függvénykönyvtár használatához a fejlesztőnek szüksége van egy C fordítóra, illetve rendelkeznie kell \cite{OpenGL}{OpenGL} és \cite{SDL2}{SDL2} függvénykönyvtárakkal is. A fejlesztő döntheti el, hogy mikor és milyen modellekre alkalmazza a vizsgálatot, illetve milyen eseményekkel jár az ütközés, ezzel személyre szabható a program minden része.\\

Az ütközésvizsgálat egyszerűen elvégezhető a check\_collision függvény meghívásával, illetve a szükséges adatok megadásával. A függvény eredményeként visszatér egy igaz, vagy hamis értékkel, ezzel jelezve, hogy a modellek ütköztek-e egymással, vagy sem.\\

A jövőben számos további fejleszti lehetőség érhető el a program szempontjából, ilyen lehet például az optimalizálás, amellyel a teljesítményt növelhetjük, míg az erőforrások nagyságát csökkenthetjük, valamint még könnyebbé, felhasználóbarátabbá tehetjük a program használatát.\\

A függvénykönyvtár széles körben alkalmazható fizikai szimulációk, videójátékok esetében, ezzel hozzájárulva a fejlesztők munkájához, illetve a számítógépes grafikában rejlő lehetőségek kiaknázásához.