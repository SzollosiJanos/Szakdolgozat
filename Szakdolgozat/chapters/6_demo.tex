
\Chapter{Demo program működtetése}

A függvénykönyvtár teszteléséhez készült egy program, amellyel szemléltethetjük a számításokat, a számítások működését és azok hatékonyságát. A teszt programban egy úgynevezett "playermodell"-t irányíthatunk, amely egy egyszerű kocka modell. A playermodellel körbejárhatjuk a teret, a modellt forgathatjuk, tükrözhetjük, méretezhetjük, illetve más modellekkel ütközhetünk. Ezen lépésekhez különböző billentyűkombinációk érhetőek el.
\begin{table}[h]
	\centering
	\begin{tabular}{|l|l|}
		\hline
		\multicolumn{1}{|c|}{\textbf{Billentyű}} & \multicolumn{1}{c|}{\textbf{Esemény}}\\
		\hline
		 W,A,S,D & {A "playermodell" mozgatása a térben X,Y tengelyen.} \\\hline
		 CTRL, SPACE & A "playermodell" mozgatása a térben Z tengelyen. \\\hline
		 Egér & A kamera mozgatása a "playermodell" körül. \\\hline
		 Görgő & A kamera távolságának állítása a "playermodellhez" képest. \\\hline
		 E & Új modell létrehozása az ütközésvizsgálat teszteléséhez. \\\hline
		 Q & A "playermodell", illetve a "hitbox" ki/be kapcsolása. \\\hline
		 ESC & Kilépés a programból. \\\hline
		 N, M & A "playermodell" méretének növelése/csökkentése. \\\hline
		 C, V, B & A "playermodell" tükrözése X, Y és Z tengelyekre. \\\hline
		 J, L & A "playermodell" forgatása az X tengelyen. \\\hline
		 I, K & A "playermodell" forgatása az Y tengelyen. \\\hline
		 U, O & A "playermodell" forgatása az Z tengelyen. \\\hline
	\end{tabular}
\end{table}

\newpage