\Chapter{Optimalizálás}

Minél több a "playermodel"\textsuperscript{\ref{def:playermodel}} háromszögeinek száma, illetve a többi modellek száma, és azok háromszögeinek száma, annál több számítást kell végeznünk. Például 2 modell esetén, ha a playermodellünk\textsuperscript{\ref{def:playermodel}} 100 háromszögből áll, és a másik modellünk is szintén 100 háromszögből áll, akkor a programnak frame-enként 100$\cdot$100, azaz $10\,000$ számítást kell végeznie, négyzetes idejű az algoritmus számítási ideje. Emiatt a módszer nem alkalmas valós idejű alkalmazásokhoz. Több módszert is kipróbáltam lehetséges optimalizálás szempontjából.

\Section{Szűrés távolság alapján}
Minden modell betöltésekkor kiszámítjuk a modell középpontjától számított legtávolabbi pontjának hosszát, ahogy a \ref{fig:opt_1} ábrán láthatjuk. Ezt csak egyszer kell kiszámítanunk, kivéve ha futtatás közben szeretnénk módosítani a modellt, például a modell méretének megváltoztatásával. Ütközésvizsgálat előtt kiszámítjuk, hogy a két modell ütközhet-e egymással vagy sem. Ez lényegében a négyzetes elválasztás(\ref{chap:quad}) módszere alapján működik.
\\
Részlet a check\_collision függvényből:
\begin{cpp}
float limit, distance;
limit = model1->farestpoint + model2->farestpoint + 0.5;
distance = get_distance(model1_position, model2_position);
if (fminf(distance, limit) == limit)
{
    return false;
}
\end{cpp}

\vfill
\begin{definition}[Playermodel]
	A "playermodel" egy számítógépes játékban használt grafikus megjelenítése a játékos karakternek, amely magában foglalja annak kinézetét és animációit a játék világában való mozgása során.
	\label{def:playermodel}
\end{definition}
\newpage
A limit változóban megnézzük a maximális távolságot, amelyen belül már ütközésvizsgálatot vizsgálunk. Összeadjuk a két modell legtávolabbi pontjának távolságát, illetve egy extra konstanst. Ezután a distance változóba elmentjük a két modell közötti távolságot. Ha a limit kisebb, mint a distance, akkor nem lehetséges a két modell ütközése, azért visszatérünk false értékkel, nem vizsgálunk ütközést.

Az optimalizálás ezen esetben \textbf{sikeres}, kevesebb számítást veszünk figyelembe, a program gyorsul, amely függ a modellek távolságától, méretétől és komplexitásától.
\begin{figure}[h]
	\centering
	\includegraphics[width=13truecm, height=7.5truecm]{images/opt_5.1.png}
	\caption{Modell köré egy gömb alakú "hitbox"\textsuperscript{\ref{def:hitbox}} generálása}
	\label{fig:opt_1}
\end{figure}

\newpage
\Section{Távolság szűrése háromszögekkel}
\label{opt_tav}
Lényege, hogy ütközésvizsgálat előtt a modellek háromszögeinek a középpontját kiszámítjuk, majd a középpontokat viszonyítjuk egymáshoz, ahogy a \ref{fig:opt_2} ábrán láthatjuk. Ha a háromszögek közötti távolság nagyobb, mint a maximum táv, akkor az adott háromszög számítása elhanyagolható, mivel biztosra vehetjük, hogy nem lesz az adott két háromszög esetén metszés.

Az optimalizálás ezen esetben sikertelen, a számítások ideje drasztikusan megnő, használata nem ajánlott.

\Section{Távolság szűrése háromszögek mentésével}
Lényege ugyan az, mint az előző szekcióban bemutatott algoritmusnak, annyi kivétellel, hogy a középpontokat csak a modell beolvasásakor számítjuk ki, majd lementjük későbbi használatra, ahogy a \ref{fig:opt_2} ábrán láthatjuk.

Az optimalizálás ezen esetben sikertelen, a számítások ideje megnő az optimalizálatlan programhoz képest, de csökken az \ref{opt_tav} algoritmushoz képest, illetve 25\%-al megnő a memóriaigény is. Használata nem ajánlott.
\begin{figure}[h]
	\centering
	\includegraphics[width=13truecm, height=7.5truecm]{images/opt_5.2.png}
	\caption{Távolság szűrése háromszögek középpontjával}
	\label{fig:opt_2}
\end{figure}

\newpage
\Section{Pozíció alapján szűrés}
Lényege, hogy az adott modell nem feltétlenül van mindig ugyan abban a síkban, mint a másik vizsgálandó modell. Megnézzük a két modell középpontjának koordinátáit, illetve a legtávolabbi pontját az adott tengelyhez képest, ahogy az \ref{fig:opt_3} ábrán láthatjuk. Például, ha a playermodellünk a Z koordinátán 20 pixel magas, akkor a maximum szint a Z tengelyen az a modell középpontja + 20 pixel lesz, minden más háromszöget az adott modellben figyelmen kívül hagyunk. Ezeket a lépéseket meg kell ismételnünk minden tengelyre 2 alkalommal. Ez összesen 6 különböző számítást jelent.

Az optimalizálás ezen esetben megoldható, a számítások ideje csökken, de cserébe a memória igény nő.
\begin{figure}[h]
	\centering
	\includegraphics[width=13truecm, height=7.5truecm]{images/opt_5.3.png}
	\caption{Tengelyek alapján szűrés}
	\label{fig:opt_3}
\end{figure}