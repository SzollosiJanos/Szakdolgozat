
\Chapter{Demo program működtetése}

A függvénykönyvtár teszteléséhez készítettem egy programot, amellyel szemléltethetjük a számításokat, a számítások működését és azok hatékonyságát. A teszt programban egy úgynevezett "playermodel"\textsuperscript{\ref{def:playermodel}}-t irányíthatunk, amely egy egyszerű kocka modell. A playermodellel\textsuperscript{\ref{def:playermodel}} körbejárhatjuk a teret, a modellt forgathatjuk, tükrözhetjük, méretezhetjük, illetve más modellekkel ütközhetünk. A program működtetéséhez használható különböző billentyűkombinációkat a \ref{tab:demo} táblázat szemlélteti.
\begin{table}[h]
	\centering
	\begin{tabular}{|l|l|}
		\hline
		\multicolumn{1}{|c|}{\textbf{Billentyű}} & \multicolumn{1}{c|}{\textbf{Esemény}}\\
		\hline
		 W,A,S,D & {A "playermodel"\textsuperscript{\ref{def:playermodel}} mozgatása a térben X,Y tengelyen.} \\\hline
		 CTRL, SPACE & A "playermodel"\textsuperscript{\ref{def:playermodel}} mozgatása a térben Z tengelyen. \\\hline
		 Egér & A kamera mozgatása a "playermodel"\textsuperscript{\ref{def:playermodel}} körül. \\\hline
		 Görgő & A kamera távolságának állítása a "playermodelhez"\textsuperscript{\ref{def:playermodel}} képest. \\\hline
		 E & Új modell létrehozása az ütközésvizsgálat teszteléséhez. \\\hline
		 Q & A "playermodel"\textsuperscript{\ref{def:playermodel}}, illetve a "hitbox"\textsuperscript{\ref{def:hitbox}} ki/be kapcsolása. \\\hline
		 ESC & Kilépés a programból. \\\hline
		 N, M & A "playermodel"\textsuperscript{\ref{def:playermodel}} méretének növelése/csökkentése. \\\hline
		 C, V, B & A "playermodel"\textsuperscript{\ref{def:playermodel}} tükrözése X, Y és Z tengelyekre. \\\hline
		 J, L & A "playermodel"\textsuperscript{\ref{def:playermodel}} forgatása az X tengelyen. \\\hline
		 I, K & A "playermodel"\textsuperscript{\ref{def:playermodel}} forgatása az Y tengelyen. \\\hline
		 U, O & A "playermodel"\textsuperscript{\ref{def:playermodel}} forgatása az Z tengelyen. \\\hline
	\end{tabular}
	\caption{Demo program használata}
	\label{tab:demo}
\end{table}

\newpage